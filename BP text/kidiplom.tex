%%%  Ukázkový text a dokumentace stylu pro text závěrečné (bakalářské a
%%%  diplomové) práce na KI PřF UP v Olomouci
%%%  Copyright (C) 2012 Martin Rotter, <rotter.martinos@gmail.com>
%%%  Copyright (C) 2014 Jan Outrata, <jan.outrata@upol.cz>



%%  V případě použití programu BibLaTeX pro tvorbu seznamu literatury
%%  je poté ještě třeba spustit program Biber s parametrem jméno
%%  souboru zdrojového textu bez přípony a následně opět (dvakrát)
%%  přeložit zdrojový text programem pdfLaTeX.

%%  Postup získání Postscriptového souboru je popsán v dokumentaci.


%%  Třída dokumentu implementující styl pro závěrečnou práci. Vybrané
%%  nepovinné parametry (ostatní v dokumentaci):

%%  'master' pro sazbu diplomové práce, jinak se sází bakalářská práce

%%  'field=kód' pro Váš studijní obor, kódy pro diplomovou práci 'uvt'
%%  pro Učitelství výpočetní techniky pro střední školy a 'binf' pro
%%  Bioinformatiku, jinak je výchozí Informatika, a pro bakalářskou
%%  práci 'ainfk' pro Aplikovanou informatiku v kombinované formě,
%%  'inf' pro Informatiku, 'infv' pro Informatiku pro vzdělávání a
%%  'binf' pro Bioinfomatiku, jinak je výchozí Aplikovaná informatika
%%  v prezenční formě

%%  'printversion' pro sazbu verze pro tisk (nebarevné logo a odkazy,
%%  odkazy s uvedením adresy za odkazem, ne odkazy do rejstříku),
%%  jinak verze pro prohlížeč

%%  'biblatex' pro zapnutí podpory pro sazbu bibliografie pomocí
%%  BibLaTeXu, jinak je výchozí sazba v prostředí thebibliography

%%  'language=jazyk' pro jazyk práce, jazyky english pro anglický,
%%  slovak pro slovenský, jinak je výchozí czech pro český

%%  'font=sans' pro bezpatkový font (Iwona Light), jinak výchozí
%%  patkový (Latin Modern)

\documentclass[
%  master,
  field=inf,
%  printversion,
  biblatex,
%  language=english,
%  font=sans,
  glossaries,
  index
]{kidiplom}

%% Informace pro úvodní strany. V jazyku práce (pokud není v komentáři
%% uvedeno česky) a anglicky. Uveďte všechny, u kterých není v
%% komentáři uvedeno, že jsou volitelné. Při neuvedení se použijí
%% výchozí texty. Text pro jiný než nastavený jazyk práce (nepovinným
%% parametrem language makra \documentclass, výchozí český) se zadává
%% použitím makra s uvedením jazyka jako nepovinného parametru.

%% Název práce, česky a anglicky. Měl by se vysázet na jeden řádek.
\title{Aplikace pro vytváření tréninkových plánů}
\title[english]{Application for creating training plans}

%% Volitelný podnázev práce, česky a anglicky. Měl by se vysázet na
%% jeden řádek. Výchozí je prázdný.
%\subtitle{Ukázkový text a dokumentace stylu v \LaTeX{}u}
%\subtitle[english]{Sample text and documentation of the \LaTeX{} style}

%% Jméno autora práce. Makro nemá nepovinný parametr pro uvedení
%% jazyka.
\author{Jiří Mlčoušek}

%% Jméno vedoucího práce (včetně titulů). Makro nemá nepovinný
%% parametr pro uvedení jazyka.
\supervisor{Mgr. Tomáš Urbanec}

%% Volitelný rok odevzdání práce. Výchozí je aktuální (kalendářní)
%% rok. Makro nemá nepovinný parametr pro uvedení jazyka.
\yearofsubmit{2024}

%% Anotace práce, včetně anglické (obvykle překlad z jazyka
%% práce). Jeden odstavec!
\annotation{Student naprogramuje webovou aplikaci pro vytváření tréninkových plánů
na míru uživateli. Uživatel bude moci zadávat zpětnou vazbu o průběhu a
plnění tréninku. Dosažené výsledky budou v aplikaci přehledně zobrazeny,
případně i s vizualizací vývoje v čase. Aplikace také bude schopna
generovat náhled aktuálního plánu pro použití bez přístupu k internetu.
Nakonec se student pokusí o hlubší analýzu dosažených výsledků a jejich
promítnutí do návrhu následujících plánů.}

\annotation[english]{The student will program a web application for creating training plans
tailored to the user. The user will be able to enter feedback on the progress and
performance of the training. The achieved results will be clearly displayed in the application,
possibly with a visualisation of the evolution over time. The application will also be able to
generate a preview of the current plan for use without internet access.
Finally, the student will attempt a deeper analysis of the results achieved and their
reflecting on the design of the following plans.}

%% Klíčová slova práce, včetně anglických. Oddělená (obvykle) středníkem.
\keywords{tréninkový plán, . . .}
\keywords[english]{training plan, . . .}

%% Volitelná specifikace příloh textu práce, i anglicky. Výchozí je '1
%% CD/DVD'.
%\supplements{jedno kulaté placaté CD/DVD s malou kulatou dírou uprostřed}
%\supplements[english]{one round flat CD/DVD with a small round hole in the middle}

%% Volitelné poděkování. Stručné! Výchozí je prázdné. Makro nemá
%% nepovinný parametr pro uvedení jazyka.
\thanks{Děkuji, děkuji, děkuji.}

%% Cesta k souboru s bibliografií pro její sazbu pomocí BibLaTeXu
%% (zvolenou nepovinným parametrem biblatex makra
%% \documentclass). Použijte pouze při této sazbě, ne při (výchozí)
%% sazbě v prostředí thebibliography.
\bibliography{bibliografie.bib}

%% Další dodatečné styly (balíky) potřebné pro sazbu vlastního textu
%% práce.
\usepackage{lipsum}

\begin{document}
%% Sazba úvodních stran -- titulní, s bibliografickými údaji, s
%% anotací a klíčovými slovy, s poděkováním a prohlášením, s obsahem a
%% se seznamy obrázků, tabulek, vět a zdrojových kódů (pokud jejich
%% sazba není vypnutá).
\maketitle

%% Vlastní text závěrečné práce. Pro povinné závěry, před přílohami,
%% použijte prostředí kiconclusions. Povinná je i příloha s obsahem
%% přiloženého CD/DVD.

%% -------------------------------------------------------------------

\newcommand{\BibLaTeX}{\textsc{Bib}\LaTeX}



\section{Úvod}
V dnešní uspěchané době se stále více lidí obrací k zdravému životnímu stylu a fyzické aktivitě. Bez ohledu na věk, pohlaví nebo úroveň zkušeností, je cvičení nezbytným prvkem udržení fyzické kondice a celkového zdraví. Nicméně, mnoho jednotlivců stojí před problémem, který může zásadně ovlivnit jejich úspěch v posilovně - absencí tréninkového plánu. Mnozí z nás touží po cvičení, ale často nemají dostatek znalostí nebo motivace, aby si sami vytvořili efektivní plán. Navíc si nechtějí najímat trenéry kvůli finančním nákladům a možným omezením spojeným s časovými a prostorovými závazky.

Tento problém může být zvláště znepokojující, neboť správně navržený tréninkový plán může výrazně zvýšit účinnost cvičení a minimalizovat riziko zranění. V tomto kontextu se nabízí otázka: Jak můžeme pomoci lidem dosáhnout svých fitness cílů a zároveň jim ušetřit čas a peníze? Odpovědí je webová aplikace, která umožní jednoduché a přístupné vytváření tréninkových plánů pro posilovnu.\\

\newacronym {TP} {TP} {Tréninkový plán} \gls{TP} \\
\noindent\textcolor{gray}{první verze, ještě to upravím později}



\subsection{Motivace}

Mým osobním zájmem a motivací pro vytvoření této bakalářské práce je spojení mého dlouholetého koníčku, kterým je cvičení, s oborem informatika, který studuji. Jsem přesvědčen, že technologie mohou hrát klíčovou roli v tom, jak pomáháme lidem dosáhnout svých cílů a zlepšit jejich životní kvalitu.

Již několik let jsem aktivním pravidelným návštěvníkem posilovny a během této doby jsem si uvědomil, jak mnoho lidí se potýká s problémem nedostatku tréninkového plánu a motivace. Chtěl bych tímto projektem vytvořit webovou aplikaci, která bude snadno použitelná pro širokou veřejnost, a která umožní uživatelům rychle a efektivně vytvářet tréninkové plány přizpůsobené jejich cílům a potřebám. Tímto způsobem bych chtěl přispět k zlepšení fyzické kondice a životního zdraví jednotlivců, kteří touží po cvičení, ale potřebují podporu a orientaci.

Tato bakalářská práce bude spojovat mé znalosti a dovednosti v oboru informatiky s mé vášní pro fitness a zdravý životní styl. Věřím, že výsledek této práce bude mít potenciál pozitivně ovlivnit mnoho lidí a pomoci jim dosáhnout svých fitness cílů bez zbytečných obtíží.\\

\noindent\textcolor{gray}{první verze, ještě to upravím později}

\subsection{Specifikace zadání}
Aplikace by měla mít následující funkce:

\begin{itemize}

\item vytvoření tréninkového plánu na míru
\subitem vytvoření účtu
\subitem zadání informací a cíle

\item zaznamenávání zpětné vazby uživatelem

\item zobrazení dosažených výsledků

\item možnost vygenerování náhledu \gls{TP}

\item vygenerovat následující plán na základě zadaných dat

\end{itemize}

\subsubsection{Vytvoření \gls{TP} na míru}
Uživatel si vytvoří klasický účet pomocí přihlašovacího jména a hesla. Bude mít možnost klasické správy svého účtu. Součástí tvorby účtu bude i zadání aktuálních informací o sobě. Budou zda jak informace stálé, např. věk, pohlaví, výška . . ., ale i informace které by se měly pravidelně aktualizovat (jednou za určité časové období vyskočí uživateli při otevření aplikace stránka s aktualizací) jako např. váha, pocit z plánu, nálada . . . 

\subsubsection{Zaznamenávání zpětné vazby uživatelem}
Uživatel poté po dokončení tréninku zadá zda splnil/nesplnil trénink. Potvrdí zda zvedl předepsané váhy . . . 
\noindent\textcolor{gray}{ještě musím popřemýšlet, jak si to přesně představuji, aby to bylo dobré a zároveň zrealizovatelné}


\subsubsection{Zobrazení dosažených výsledků}
Uživatel bude mít možnost se dostat do sekce s dosaženými výsledky, kde bude \clqq síň slávy\crqq tj. postupně vypsané \newacronym {PR} {PR} {Personal record - maximálka} \gls{PR} na všechny základní cviky (bench press, squad \& deadlift) a zároveň vypsané i dosažené hodnoty na ostatní cviky. Dále zde budou statistiky např. počet nazvedané váhy nebo počet opakování jako souhrny. Určité hodnoty budou znázorněny i graficky.

\subsubsection{Vygenerování náhledu \gls{TP}}
Uživatel bude mít možnost si buď celý nebo jen určitou část \gls{TP} vygenerovat formy .jpeg, tím pádem bude moct mít \gls{TP} pořád u sebe i offline.

\subsubsection{Vygenerování následujícího plánu na základě zadaných dat} 


%%%%%%%%%%%%%%%%%%%%%%%%%%
\section{Již existující podobné aplikace}
Na trhu již existuje několik aplikací zaměřených na plánování cvičení, tvorbu tréninkových plánů a podporu zdravého životního stylu. Tyto aplikace se snaží naplňovat potřeby uživatelů, kteří hledají nástroje pro efektivní cvičení a dosahování svých fitness cílů. Většinou poskytují možnosti monitorování pokroku, návody na cvičení, a některé i tréninkové plány.

I když tyto existující aplikace nabízejí užitečné funkce pro plánování a sledování cvičení, stále může být pro některé jednotlivce obtížné najít aplikaci, která by plně vyhovovala jejich potřebám a preferencím. Někdy mohou být komplikované na použití, neobsahovat dostatečně přizpůsobitelné tréninkové plány nebo sloužit pouze pro domácí použití a nevyužijeme ji tak v posilovně. To ukazuje na stále existující potřebu více přístupného a snadno použitelného nástroje pro tvorbu a plánování cvičebních režimů.

\subsection{Fytify}
Fytify je mobilní aplikace, kterou jsem sám aktivně používal přibližně rok, když jsem se věnoval cvičení doma.
\noindent\textcolor{gray}{aplikaci prozkoumám a poté doplním}

\subsection{Setgraph}
Monitoring vah, nikdy jsem nepoužíval,ale zaujala mě dříve
\noindent\textcolor{gray}{aplikaci prozkoumám a poté doplním}

\subsection{Gymshark training}
Na aplikaci jsem narazil během vyhledávání informací k mé BP, vypadá dost slibně.
\noindent\textcolor{gray}{aplikaci prozkoumám a poté doplním}

\section{Použité technologie}
Mám v plánu používat \csharp  zároveň využít ASP.NET MVC

\subsection{\csharp}

\csharp \hspace{1mm} (vyslovováno jako "C-sharp") je moderní a objektově orientovaný programovací jazyk vyvinutý společností Microsoft. 

\subsubsection{ASP.NET}

ASP.NET (Active Server Pages .NET) je webový framework vyvinutý společností Microsoft. Jedná se o technologii, která umožňuje vývojářům vytvářet webové aplikace a webové stránky s využitím programovacího jazyka \csharp \hspace{1mm} nebo VB.NET (Visual Basic .NET). ASP.NET poskytuje robustní a výkonný způsob tvorby dynamických a interaktivních webových aplikací.


\subsubsubsection{MVC model}
Model v MVC architektuře ASP.NET reprezentuje datovou část aplikace. 

\subsection{MySQL}

MySQL je open-source relační databázový systém, který je široce používán pro ukládání, správu a manipulaci s daty v různých typech aplikací, od webových stránek po podnikové aplikace.

\subsection{HTML \& CSS}
Společně HTML a CSS tvoří základní stavební kameny webových stránek.

\subsubsection{HTML}

HTML (HyperText Markup Language) je základní značkovací jazyk používaný pro tvorbu webových stránek.

\subsubsection{CSS}

CSS (Cascading Style Sheets) je stylizační jazyk používaný k definici vzhledu a prezentace webových stránek v kombinaci s HTML. 

\section{Tréninkové plány}
Základní popis co to je \gls{TP} jak se používá a jak to funguje

\subsection{Tréninkové splity}
Co to je a jak to funguje

\subsubsection{Full Body}
Pro za

\subsubsection{Body Parts}
R;zn0 partie

\newacronym {PPL} {PPL} {Push Pull Legs} 

\subsubsection{\gls{PPL}}
rozepsat

\subsection{Základní cviky}
rozepisu partie a ke kezde par cviku a info

\subsubsection{Nohy}
\subsubsection{Záda}
\subsubsection{Prsa}
\subsubsection{Ramena}
\subsubsection{Ruce}
\subsubsubsection{Biceps}
\subsubsubsection{Triceps}
\subsubsection{Core}



\section{Programátorská dokumentace}

\subsection{Struktura projektu}

\subsection{Databáze}

\section{Uživatelská dokumentace}

\subsection{Současný plán}

\subsection{\clqq Síň slávy\crqq}

\section{Potenciální rozšíření aplikace}

\appendix

\section{První příloha}
Text první přílohy

\section{Obsah přiloženého CD/DVD} \label{sec:ObsahCD}


%% -------------------------------------------------------------------

%% Sazba volitelného seznamu zkratek, za přílohami.
\printglossary

%% Sazba povinné bibliografie, za přílohami (případně i za seznamem
%% zkratek). Při použití BibLaTeXu použijte makro
%% \printbibliography. jinak prostředí thebibliography. Ne obojí!

%% Sazba i v textu necitovaných zdrojů, při použití
%% BibLaTeXu. Volitelné.
\nocite{*}
%% Vlastní sazba bibliografie při použití BibLaTeXu.
\printbibliography

%% Bibliografie, včetně sazby, při nepoužití BibLaTeXu.
 \begin{thebibliography}{9}

%\bibitem{kniha2} \uppercase{Hawke}, Paul. NanoHttpd: Light-weight HTTP server designed for embedding in other applications. GitHub [online]. 2014-05-12. [cit. 2014-12-06]. Dostupné z: \url{https://github.com/NanoHttpd/nanohttpd}


 \end{thebibliography}

%% Sazba volitelného rejstříku, za bibliografií.
\printindex

\end{document}
